\documentclass{article}
\usepackage[utf8]{inputenc}

%% Theorems

\usepackage{amsthm}
\usepackage{thmtools}

\declaretheorem[style=plain,numberwithin=section]{theorem}
\declaretheorem[style=plain,numbered=no,name=Theorem]{theorem*}
\declaretheorem[style=plain,sibling=theorem]{proposition}
\declaretheorem[style=plain,sibling=theorem]{lemma}
\declaretheorem[style=plain,sibling=theorem]{corollary}

\declaretheorem[style=definition,numberwithin=section]{definition}
\declaretheorem[style=definition,qed=$\diamondsuit$,numberwithin=section]{example}
\declaretheorem[style=definition,qed=$\triangle$,numberwithin=section]{remark}



\title{Math}
\author{birger.moell }
\date{February-August 2018}

\begin{document}

\maketitle

\section{Inledning}
Den här boken är en översikt och sammanfattning av matematiken som jag lär mig.
\maketitle

\section{Set theory}

\begin{definition}[Set]
A set is an unorder collection of elements.
\end{definition}

\begin{definition}[Powerset]
The powerset of a set $A$ is ...
If S is the set {x, y, z}, then the subsets of S are
{} also denoted O/, the empty set or the null set
{x}
{y}
{z}
{x, y}
{x, z}
{y, z}
{x, y, z}
All the previous sets containing the set including itself.

\end{definition}


\textbf{Exempel 1:}
\newline
Mängden som innehåller siffrorna (objekten) 1,2 3 skriver vi
\{ 1,2,3 \}
\newline
\textbf{Den tomma mängden}
\{\}
\newline
\textbf{Exempel 2:}
\newline

\begin{equation}
  Elementet  \alpha  ligger i mängden A
\end{equation}

\section{Cartesian product}


\begin{definition}[Cartesian product]
möngd = set
In set theory, a Cartesian product is a mathematical product is a mathematical operation
that returns a set from multiple sets.

For sets A and B, the Cartesian product A x B is the set of all ordered pairs (a, b) where
a $C$ $A$ and b $C$ $B$.
Products can be specified using set-builder notation.




\begin{equation}
  Elementet  \alpha  ligger i mängden A
\end{equation}

\section{Binary relation}


\begin{definition}[Binary relation]
A binary relation $R$ on a set $A$ is a set of ordered pairs of elements of $A$

Examples


\end{definition}

Binary relations

\begin{theorem}
  A transitive relation is "multitransitive".
\end{theorem}

\begin{proof}
  waat
\end{proof}

\section{Funktioner}

\section{Trigonometri}

\section{Vektorer}

\section{Matriser}

\section{Vektorgeometri}

\section{Lineär algebra}


\section{Math of machine learning}

\section{Reinforcement Learning}
Reinforcement learning is modeled as a Markov decision
process:
- a set of environment and agent states, S
- a set of actions, A,
of the agent

Pa(s,s') = Pr(st+1 = s'|st = s' at = a)



$\sum$

$$\sum_{n=1}^{\infty} 2^{-n} = 1$$

$\int_{a}^{b} x^2 dx$

\(x^2 + y^2 = z^2\)

\(5 + 2 = 7\)

\(x2 + 5 = 10\)



\end{document}

